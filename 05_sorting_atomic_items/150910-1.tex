\exercise

Consider the Snow Plow technique (\autoref{alg:snowplow}):
%
\begin{enumerate}

  \item Prove that the average length of the formed runs is $2M$, where $M$ is
  the size of the available internal memory.

  \item Compute the average run length by assuming that the probability to
  insert an item in the in-memory-heap is $\sfrac{2}{3}$ (instead of the classic
  $\sfrac{1}{2}$).

\end{enumerate}

\solution

\begin{enumerate}

  \item When the Snow Plow terminates, $\mathcal{H} = \emptyset$ and
  $|\mathcal{U}| = M$. Assuming $\tau$ elements have been read from the input
  sequence, we can state that $M$ of these elements ended in $\mathcal{U}$ while
  the remaining $\tau - M$ ended in $\mathcal{H}$. So the length of the sorted
  run is $M + (\tau - M) = \tau$, where the first $M$ is the number of elements
  that were present in $\mathcal{H}$ before the beginning of the procedure,
  while $\tau - M$ the number of elements inserted from the input sequence.
  Assuming a uniformly distributed input sequence, we can say that for each
  $min$ retrieved from $\mathcal{H}$, the probability that the next element is
  greater than $min$ is $\sfrac{1}{2}$. Therefore, the input items are equally
  distributed between $\mathcal{H}$ and $\mathcal{U}$, and we can derive the
  conclusion $\frac{1}{2}\tau = |\mathcal{H}| = |\mathcal{U}| = M \implies \tau
  = 2M$.

  \item $\frac{2}{3}\tau = |\mathcal{H}| \implies \frac{1}{3}\tau =
  |\mathcal{U}| = M \implies \tau = 3M$.

\end{enumerate}
