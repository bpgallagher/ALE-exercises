\exercise

\begin{enumerate}

  \item Show the pseudo-code of the variant of binary quicksort that reduces the
  extra-space used during recursive calls, and comments on its complexity and
  correctness.

  \item State and prove the theorem that establishes the partitioning
  performance of the multi-pivot selection strategy.

\end{enumerate}

\solution

\begin{enumerate}

  \item \autoref{alg:boundedqs} reduces the extra space used during recursive
  calls by substituting the recursive call on the larger part of the input with
  another execution of the body of the while. By doing this, ensures that the
  number of recursive calls is no more than $O(\log_2 n)$. Moreover, if the
  input is small enough (hence the $j - i > n_0$ in the while condition), this
  algorithm executes an insertion sort instead, which performs well on very
  small sequences. The correctness is straightforward, since whenever the
  algorithm performs a recursive call on the small subsequence it also changes
  the $i$ and $j$ such that they will delimit the larger part, which will be
  processed by the next iterative loop.
  %
  \begin{algorithm}[tb]
  \caption{Bounded Quicksort}\label{alg:boundedqs}
  \begin{algorithmic}[1]
  \Procedure{BoundedQS}{$S$, $i$, $j$}
    \While{$j - i > n_0$}
      \State $r \gets$ pick the position of a ``good pivot''
      \State swap $S[r]$ with $S[i]$
      \State $p \gets$ \Call{Partition}{$S$, $i$, $j$}
      \If{$p \le \frac{i+j}{2}$}
        \State \Call{BoundedQS}{$S$, $i$, $p - 1$}
        \State $i \gets p + 1$
      \Else
        \State \Call{BoundedQS}{$S$, $p + 1$, $j$}
        \State $j \gets p - 1$
      \EndIf
    \EndWhile
    \State \Call{InsertionSort}{$S$, $i$, $j$}
  \EndProcedure
  \end{algorithmic}
  \end{algorithm}

  \item \begin{description}

    \item[Lemma.] Let $k \ge 2$ and $a + 1 = 12 \ln k$. A sample f size $(a +
    1)k - 1$ suffices to ensure that all buckets receive less $4n/k$
    elements with probability at least $1/2$.

    \item[Proof.] Let $S'$ the sorted version of the input sequence $S$. We
    split $S'$ in $k/2$ segments of length $2n/k$ each, and we search for a
    bucket $B_i$ that has at least $4n/k$ elements assigned to it. Such bucket
    must have its pivots $s_{i-1}$ and $s_i$ positioned outside one of the
    previous segments. By the definition of oversampling, this may happen only
    if less than $a + 1$ samples fall in the segment $t_j$ overlapped by $B_i$.
    We can formulate this probability as
    %
    \begin{align*}
      \mathbb{P}\left[\exists B_i.\ |B_i| \ge \frac{4n}{k} \right] &\le
      \mathbb{P}\left[\exists t_j.\ t_j \text{ contains} < a + 1 \text{ samples}
      \right] && \\ &\le \frac{k}{2}\cdot\mathbb{P}\left[ \text{a specific
      segment contains} < a + 1 \text{ samples} \right] && (\text{\it union
      bound})\ .
    \end{align*}
    %
    The probability that a delimiter ends in a given segment is
    $\frac{2n}{k}\cdot\frac{1}{n} = \frac{2}{k}$. Given $X$ the random variable
    counting those samples, and assuming that $a \ge 1$ and $k \ge 2$, we have
    that
    %
    \begin{align*}
      && \mathbb{E}\left[X\right] &= \left(\left(a + 1 \right)k -
      - 1\right)\cdot \frac{2}{k} = 2\left(a + 1 \right) - \frac{2}{k} \ge
      2\left(a + 1 \right) - 1 \ge \frac{3}{2}\left(a + 1 \right) \\ \equiv\
      && \frac{2}{3}\cdot\mathbb{E}\left[X\right] &= \left(1 -
      \frac{1}{3}\right)\mathbb{E}\left[X\right] \ge a + 1\ .
    \end{align*}
    %
    We can now exploit the Chernoff bound, i.e. $$\mathbb{P}\left[ X < (1 -
    \delta)\mathbb{E}[X] \right] \le e^{-\frac{\delta^2}{2}\mathbb{E}[X]}\ .$$
    By setting $\delta = 1/3$ and assuming $a + 1 = 12\ln k$, we derive
    %
    \begin{align*}
      \mathbb{P}\left[X < a + 1 \right] &\le \mathbb{P}\left[X < \left(1 -
      \frac{1}{3}\right)\mathbb{E}\left[X\right] \right] \le
      \exp\left({-\frac{1}{3^2}\cdot\frac{\mathbb{E}\left[X\right]}{2}}\right) =
      \exp\left({-\frac{\mathbb{E}\left[X\right]}{18}}\right) \\ &\le
      \exp\left({-\frac{3}{2}\cdot\frac{a + 1}{18}}\right) = \exp\left({-\frac{a
      + 1}{12}}\right) = \exp(-\ln k) = \frac{1}{k}\ .
    \end{align*}
    %
    We can now conclude that
    %
    \begin{align*}      
      \mathbb{P}\left[\forall B_i.\ |B_i| < \frac{4n}{k} \right] = 1 -
      \mathbb{P}\left[\exists B_i.\ |B_i| \ge \frac{4n}{k} \right] \ge 1 -
      \frac{k}{2}\cdot\mathbb{P}\left[X < a + 1 \right] \ge 1 -
      \frac{k}{2}\frac{1}{k} = \frac{1}{2}\ .
    \end{align*}

  \end{description}

\end{enumerate}
