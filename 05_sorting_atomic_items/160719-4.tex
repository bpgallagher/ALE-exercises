\exercise

Given a sequence $S$ of $n$ elements, possibly unordered,
%
\begin{enumerate}

  \item write the pseudo-code of a randomized algorithm that receives in input
  $S$ and an integer $k$, and selects fast the $k$-th ranked item of $S$;

  \item evaluate the average time complexity of your algorithm (\emph{hint}: the
  average time complexity of your algorithm should be linear in $n$).

\end{enumerate}

\solution

\begin{enumerate}

  \item \autoref{alg:randselect} selects the $k$-th item from a given sequence
  $S$.
  %
  \begin{algorithm}[t]
  \caption{RandSelect algorithm}\label{alg:randselect}
  \begin{algorithmic}[1]
  \Function{RandSelect}{$S$, $k$}
    \State $r \gets $ \Call{Random}{1, $n$}
    \State $S_< \gets \left\{ s \in S \mid s < S[r] \right\}$
    \State $S_> \gets \left\{ s \in S \mid s > S[r] \right\}$
    \State $n_< \gets |S_<|$
    \State $n_= \gets |S| - (|S_<| + |S_>|)$
    \If{$k \le n_<$}
      \State \Return \Call{RandSelect}{$S_<$, $k$}
    \ElsIf{$k \le n_< + n_=$}
      \State \Return $S[r]$
    \Else
      \State \Return \Call{RandSelect}{$S_>$, $k - n_< - n_=$}
    \EndIf
  \EndFunction
  \end{algorithmic}
  \end{algorithm}

  \item Let us consider two (mutual exclusive) events:
  %
  \begin{itemize}

    \item $S[r]$ has rank in the range $\left[\frac{1}{3}n,
    \frac{2}{3}n\right]$. This may happen with probability $\frac{1}{3}$ and
    causes both $n_<$ and $n_>$ to be less or equal than $\frac{2}{3}n$.

    \item $S[r]$ has rank outside the previous range. This may happen with
    probability $\frac{2}{3}$.

  \end{itemize}
  %
  Since the recursive calls in \autoref{alg:randselect} are mutual exclusive, we
  may formulate its average time as
  %
  \begin{align*}
    && && &&        && \widehat{T}(n) &\le O(n) + \frac{1}{3}\cdot\widehat{T}\left(\frac{2}{3}n\right) + \frac{2}{3}\cdot\widehat{T}\left(n\right) && && && \\
    && && && \equiv && \frac{1}{3}\cdot\widehat{T}\left(n\right) &\le O(n) + \frac{1}{3}\cdot\widehat{T}\left(\frac{2}{3}n\right) && && && \\
    && && && \equiv && \widehat{T}\left(n\right) &\le O(n) + \widehat{T}\left(\frac{2}{3}n\right) = O(n)\ ,&& && &&
  \end{align*}
  %
  where the first term at the right-hand side of the first equation is the
  complexity of the partitioning of $S$ in $S_<$ and $S_>$, which can be
  computed in linear time (and also linear I/Os). The second term represents the
  time complexity of the first event, multiplied by its probability, while the
  last term its complementary event, for which we assigned a crude upper bound.
  The last equation is the result of the application of the master theorem.

\end{enumerate}
