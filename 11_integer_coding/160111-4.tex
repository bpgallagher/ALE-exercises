\exercise

Given a sequence of integers $S = (1, 6, 15, 18, 21, 24, 30)$, encode them
using:
%
\begin{itemize}
  \item Delta encoding;
  \item Rice code with $k = 3$;
  \item Elias-Fano encoding.
\end{itemize}

\solution

Since the sequence $S$ is \emph{increasing}, to achieve a shorter representation
with $\gamma$, $\delta$ and Rice coding, we preprocess the sequence using GAP
encoding, obtaining the sequence $S' = (1, 5, 9, 3, 3, 3, 6)$. The codes are
shown in \autoref{tab:codings}.
%
\begin{table}[b]
  \centering
  \begin{tabular}{c|c||r@{\hskip 5pt}l|r@{\hskip 5pt}l|r@{\hskip 5pt}l||r@{\hskip 5pt}l}
    $S$ & $S'$ & \multicolumn{2}{c|}{$\gamma(S')$} & \multicolumn{2}{c|}{$\delta(S')$} & \multicolumn{2}{c||}{$R_3(S')$} & \multicolumn{2}{c}{$(S)_2$} \\\hline
    1  & 1 & 1 &        & 1 &         & 1 & 000  & 00 & 001 \\
    6  & 5 & 001 & 01   & 011 & 01    & 1 & 100  & 00 & 110 \\
    15 & 9 & 0001 & 001 & 00100 & 001 & 01 & 000 & 01 & 111 \\
    18 & 3 & 01 & 1     & 010 & 1     & 1 & 010  & 10 & 010 \\
    21 & 3 & 01 & 1     & 010 & 1     & 1 & 010  & 10 & 101 \\
    24 & 3 & 01 & 1     & 010 & 1     & 1 & 010  & 11 & 000 \\
    30 & 6 & 001 & 10   & 011 & 10    & 1 & 101  & 11 & 110 \\
  \end{tabular}

  \caption{The $\gamma$, $\delta$ and Rice coding for the sequence $S'$ and the
  binary representation of the sequence $S$ using $b = 5$ bits.}
  \label{tab:codings}

\end{table}

Since Elias-Fano need an increasing sequence, we need to encode the original $S$
using $b = \lceil \log_2 S_n \rceil = 5$, (with maximum value $u = 2^b = 32$).
The resulting binary representation of the elements are shown in
\autoref{tab:codings}. The array $L$ will store the $\ell = \lceil\log_2
u/n\rceil = 3$ least significant digits of each representation, i.e., $$L = 001\
110\ 111\ 010\ 101\ 000\ 110\ .$$ The array $H$ will store instead the number of
occurrences of every string of $b - \ell$ bits in the most significant digits
represented in inverse unary code, i.e., $$H = \underbracket{110}_{00}\
\underbracket{10}_{01}\ \underbracket{110}_{10}\ \underbracket{10}_{11}\ .$$ To
access the $i$-th item we use a $\text{Select}_1$ data structure over $H$, and
the array $L$, as follows
%
\begin{enumerate}

  \item We compute the upper part of the binary representation with $h =
  (\text{Select}_1(i, H) - i)_2$;

  \item We retrieve the lower part with $l = L[i]$;

  \item We concatenate the two results, obtaining the original
  representation $(S_i)_2 = h \oplus l$.

\end{enumerate}
