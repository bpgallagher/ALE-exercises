\exercise

Perform the intersection between the two sets $S_1 = [1, 8, 10]$ and $S_2 = [1,
2, 5, 7, 10, 15, 20]$ via
%
\begin{enumerate}
  \item the ``mutual partition'' algorithm;
  \item the algorithm based on ``binary search with exponential jumps''.
\end{enumerate}

\solution

\begin{enumerate}
  \item {\sc MutualPartition($[1, \underline{8}, 10]$, $[1, 2, 5, 7, 10, 15, 20]$)} $\implies$ 8 {\bf not found}
    \begin{itemize}[topsep=-0.1em]
      \item[$\hookrightarrow$] {\sc MutualPartition($[\underline{1}]$, $[1, 2, 5, 7]$)} $\implies$ {\bf output} 1
        \begin{itemize}
          \item[$\hookrightarrow$] {\sc MutualPartition($\emptyset$, $\emptyset$)} $\implies$ {\bf stop}
          \item[$\hookrightarrow$] {\sc MutualPartition($\emptyset$, $[2, 5, 7]$)} $\implies$ {\bf stop}
        \end{itemize}
      \item[$\hookrightarrow$] {\sc MutualPartition($[\underline{10}]$, $[10, 15, 20]$)} $\implies$ {\bf output} 10
        \begin{itemize}
          \item[$\hookrightarrow$] {\sc MutualPartition($\emptyset$, $\emptyset$)} $\implies$ {\bf stop}
          \item[$\hookrightarrow$] {\sc MutualPartition($\emptyset$, $[15, 20]$)} $\implies$ {\bf stop}
        \end{itemize}
    \end{itemize}

  \item \autoref{fig:doubling-intersection} shows the steps computed by the
  algorithm.
  %
  \begin{figure}[t]
    \begin{subfigure}{\linewidth}
      \centering
      \begin{tikzpicture}[
          node distance=0.8cm,
          every node/.append style={
            draw,rectangle,
            minimum width=0.8cm,
            minimum height=0.8cm,
            inner sep=0
          }
        ]

        \node[draw=none] (S1) {$S_1$};
        \node[right of=S1, very thick] (B1) {1};
        \node[right of=B1] (B2) {8};
        \node[right of=B2] (B3) {10};

        \node[draw=none, right=1.6cm of B3] (S2) {$S_2$};
        \node[very thick, right of=S2] (A0) {$\textasteriskcentered$};
        \node[right of=A0, fill=gray] (A1) {1};
        \node[right of=A1] (A2) {2};
        \node[right of=A2] (A3) {5};
        \node[right of=A3] (A4) {7};
        \node[right of=A4] (A5) {10};
        \node[right of=A5] (A6) {16};
        \node[right of=A6] (A7) {20};

        \path (A0) edge[-latex, out=90, in=90, distance=0.5cm] (A1);
        \path node[below=0.5cm of A1, draw, circle, fill=black, inner sep=0.5mm, minimum width=0, minimum height=0] {} edge[dashed, -latex, out=90, in=-90] (A1);
      \end{tikzpicture}
      \caption{}
    \end{subfigure}
    \begin{subfigure}{\linewidth}
      \centering
      \begin{tikzpicture}[
          node distance=0.8cm,
          every node/.append style={
            draw,rectangle,
            minimum width=0.8cm,
            minimum height=0.8cm,
            inner sep=0
          }
        ]

        \node[draw=none] (S1) {$S_1$};
        \node[right of=S1] (B1) {1};
        \node[right of=B1, very thick] (B2) {8};
        \node[right of=B2] (B3) {10};

        \node[draw=none, right=1.6cm of B3] (S2) {$S_2$};
        \node[right of=S2] (A0) {$\textasteriskcentered$};
        \node[very thick, right of=A0] (A1) {1};
        \node[right of=A1] (A2) {2};
        \node[right of=A2] (A3) {5};
        \node[right of=A3, fill=lightgray] (A4) {7};
        \node[right of=A4, fill=lightgray] (A5) {10};
        \node[right of=A5] (A6) {16};
        \node[right of=A6] (A7) {20};

        \path (A1) edge[-latex, out=70, in=110, distance=0.3cm] (A2);
        \path (A1) edge[-latex, out=80, in=100, distance=0.5cm] (A3);
        \path (A1) edge[-latex, out=90, in=90, distance=0.8cm] (A5);

        \path (A4) edge[-latex, out=-80, in=-100, distance=0.5cm, dashed] (A5);
        \path node[below=0.5cm of A4, draw, circle, fill=black, inner sep=0.5mm, minimum width=0, minimum height=0] {} edge[dashed, -latex, out=90, in=-90] (A4);
      \end{tikzpicture}
      \caption{}
    \end{subfigure}
    \begin{subfigure}{\linewidth}
      \centering
      \begin{tikzpicture}[
          node distance=0.8cm,
          every node/.append style={
            draw,rectangle,
            minimum width=0.8cm,
            minimum height=0.8cm,
            inner sep=0
          }
        ]

        \node[draw=none] (S1) {$S_1$};
        \node[right of=S1] (B1) {1};
        \node[right of=B1] (B2) {8};
        \node[right of=B2, very thick] (B3) {10};

        \node[draw=none, right=1.6cm of B3] (S2) {$S_2$};
        \node[right of=S2] (A0) {$\textasteriskcentered$};
        \node[right of=A0] (A1) {1};
        \node[right of=A1] (A2) {2};
        \node[right of=A2] (A3) {5};
        \node[right of=A3, very thick] (A4) {7};
        \node[right of=A4, fill=gray] (A5) {10};
        \node[right of=A5] (A6) {16};
        \node[right of=A6] (A7) {20};

        \path (A4) edge[-latex, out=90, in=90, distance=0.5cm] (A5);

        \path node[below=0.5cm of A5, draw, circle, fill=black, inner sep=0.5mm, minimum width=0, minimum height=0] {} edge[dashed, -latex, out=90, in=-90] (A5);
      \end{tikzpicture}
      \caption{}
    \end{subfigure}

    \caption{Doubling search executed on the given sequences. Solid lines
    represent the steps of the exponential jumps, while dashed lines the ones of
    the binary search, which is performed over the gray subsequence (darker cells
    represent items of the intersection). Thick nodes represent the algorithm's
    current item of each sequences: for $S_1$ is the node which is currently
    searched in the longer sequence, while for $S_2$ is the last item intersected
    or, if not found, the maximal item in $S_2$ which is lower than last searched
    item (for the first iteration, is just a placeholder).}

    \label{fig:doubling-intersection}
  \end{figure}

\end{enumerate}
