\exercise

Show and prove the two properties that allow the LCP-algorithm to run in time
linear in the string length. Comment on how they are used in that algorithm to
compute the LCP-array.

\solution

Given the suffix array $SA$ for a given string $T$, the two properties are the
following:
%
\begin{align}\label{eq:lcp-1}
  \forall x < y.\ \text{LCP}\big(\text{\it suff}_{SA[y-1]}, \text{\it
  suff}_{SA[y]}\big) \ge \text{LCP}\big(\text{\it suff}_{SA[x]}, \text{\it
  suff}_{SA[y]}\big)\ ;
\end{align}
%
\begin{align}\label{eq:lcp-2}
    &\text{LCP}\big(\text{\it suff}_{SA[y-1]}, \text{\it suff}_{SA[y]}\big) > 0
    \\\Rightarrow\ &\text{LCP}\big(\text{\it suff}_{SA[y-1] + 1}, \text{\it
    suff}_{SA[y] + 1}\big) = \text{LCP}\big(\text{\it suff}_{SA[y-1]}, \text{\it
    suff}_{SA[y]}\big) - 1\ .\nonumber
\end{align}
%
\autoref{eq:lcp-1} holds since the suffixes of $T$ are lexicographically
ordered, so as we go farther from $SA[y]$ we reduce the length of the shared
prefix, while \autoref{eq:lcp-2} holds since in the right-hand side of the
equality we are just eliding the first character of the common prefix between
the two strings in the left-hand side.

These properties are exploited in Kasai's algorithm since, scanning every suffix
from the longest to the shortest, it avoids recomputing the common prefixes
shared by $\text{\it suff}_p$, $\text{\it suff}_q$ and $\text{\it suff}_{p+1}$,
$\text{\it suff}_{q+1}$, where $p$ and $q$ are two adjacent suffixes in the
suffix array.
